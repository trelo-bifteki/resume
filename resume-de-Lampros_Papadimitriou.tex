% !TEX encoding = UTF-8
% !TEX program = pdflatex
% !TEX spellcheck = de_DE

\documentclass[german,a4paper, nologo, notitle, nototpages]{mypasscv}
\usepackage[german]{babel}

\ecvname{Lampros Papadimitriou}
\ecvaddress{Suitbertusstr. 158, 40223 Düsseldorf}
\ecvtelephone[+49 172 3990194]{}
\ecvemail{lambrospd@gmail.com}
\ecvhomepage{www.trelobifteki.com}
\ecvdateofbirth{10 Mai 1983}
\ecvnationality{Deutschland, Griechenland}
\ecvfootnote{}
\ecvpicture[width=3.8cm]{avatar.jpg}

\begin{document}
	\begin{mypasscv}
		\ecvpersonalinfo

		\ecvbigitem{Beruf}{Fullstack Software Entwickler}

		\ecvblueitem{}{
			Full-stack Software Entwickler fokussiert auf moderne Web Technologien. Interesse an Entwicklung von
			Web-Anwendungen, System-Integration und Informationstechnologien
		}

		\ecvsection{Berufserfahrung}

		\ecvtitle{July 2019 -- Heute}{Deutsche Apotheker- und Ärztebank}{}
		\ecvsubtitle{Fullstack software developer}
		\ecvitem{}{
			\begin{ecvitemize}
				\item Entwicklung von webbasierten Lösungen für das Kreditportal und interne Anwendungen
				\item Anforderungsanalyse und Design-Entwurf
				\item Design und Aufbau von Automatisierung Infrastruktur (CI/CD)
				\item Team Koordinierung in Agile basierten Projekten
				\item Berufliche Erfahrung in Entwicklung von Backend-Komponenten mit: Java, Groovy, Spring Framework, Hibernate, Oracle, Gradle, Elasticsearch, GIT
				\item Berufliche Erfahrung in Entwicklung von Automatisierung-Komponenten mit: Jenkins, Docker, Openshift, Kubernetes, Playwright
				\item Berufliche Erfahrung in Entwicklung von Frontend UI-Anwendungen mit: Typescript, VueJs, Webpack, Sass, CSS
			\end{ecvitemize}
		}

		\ecvtitle{November 2016 -- Juni 2019}{CHECK24 Vergleichsportal Baufinanzierung GmbH}{}
		\ecvsubtitle{Senior Fullstack Software Entwickler}
		\ecvitem{}{
			\begin{ecvitemize}
				\item Weiterentwicklung von webbasierten Lösungen für das Vergleichsportal und interne Anwendungen
				\item Anforderungsanalyse und Design-Entwurf
				\item Team-Koordination in Agile Projekten
				\item Berufliche Erfahrung in Entwicklung von Backend-Komponenten mit: Java, Spring Framework, Hibernate, Oracle, Gradle, Elasticsearch, GIT
				\item Berufliche Erfahrung in Entwicklung von Frontend UI-Anwendungen mit: Javascript, AngularJS, VueJs, Grunt, Gulp, Sass, CSS
			\end{ecvitemize}
		}

		\ecvtitle{Mai 2013 -- November 2016}{Tyntec GmbH}{}
		\ecvsubtitle{Java Software Entwickler}
		\ecvitem{}{
			\begin{ecvitemize}
				\item Entwicklung von webbasierten internen Anwendungen für SMS-Netzwerke
				\item Anforderungsanalyse in Agile Projekten
				\item Berufliche Erfahrung in Entwicklung von Backend-Komponenten mit: Java, J2EE, CDI, Spring Framework, Spring cloud, PostgreSQL, Elasticsearch, Python
				\item Berufliche Erfahrung und Umsetzung von UI-Anwendungen mit: Primefaces, AngularJS, CSS, Bootstrap, Material Design
			\end{ecvitemize}
		}

		\pagebreak

		\ecvtitle{April 2008 -- April 2013}{Externer IT-Berater für Ericsson Hellas}{}
		\ecvsubtitle{Services Engineer}
		\ecvitem{}{
			\begin{ecvitemize}
				\item Weiterentwicklung von webbasierten IPTV-Lösungen in Griechenland, Schweden und Israel
				\item Entwicklung von Lösungen mit J2EE und Web-Technologien für Telekommunikationsunternehmen
				\item Berufliche Erfahrung in Entwicklung von Anwendungen mit: Java, Spring Framework, J2EE, Oracle, Subversion, MySQL, HTML5, CSS, Javascript, Python, Perl
			\end{ecvitemize}
		}

		\ecvtitle{2005 -- 2006}{Piräus Universität, GR - Fakultät für Informatik}{}
		\ecvsubtitle{Website Entwickler und Designer}
		\ecvitem{}{
			\begin{ecvitemize}
				\item Entwicklung der Webseite für die Fakultät
				\item Berufliche Erfahrung mit: PHP, Joomla, MySQL, Linux
			\end{ecvitemize}
		}

		%\pagebreak

		\ecvsection{Studium}

		\ecvtitlelevel{2001--2008}{Bachelor of Science in Informatik}{ISCED 5a}
		\ecvsubtitle{Piräus Universität, Griechenland - Fakultät für Informatik}
		\ecvitem{}{
			\begin{ecvitemize}
				\item Bachelor Arbeit: Pluto - Netzwerkerkennung mit heuristischen Methoden
				\item Ehrenamtliche Arbeit als IT-Administrator im Labor
			\end{ecvitemize}
		}

		\ecvsection{Veröffentlichung}
		\ecvitem{}{Educational approach in Ancient Greek History by using Web technologies (2005)}

		%  \pagebreak

		\ecvsection{Sprachkenntnisse}
		\ecvmothertongue{Griechisch}
		\ecvlanguageheader
		\ecvlanguage{Deutsch}{C2}{C2}{C2}{C2}{C2}
		\ecvlastlanguage{Englisch}{C2}{C2}{C2}{C2}{C2}
		\ecvlanguagefooter

		\ecvsection{Kenntnisse}

		\ecvblueitem{Software Entwicklung}{
			\begin{ecvitemize}
				\item Clean-Code Prinzipien
				\item Java, Spring Framework, J2EE, Python, PHP, Perl
				\item Javascript, HTML, CSS, Sass, AngularJS, VueJS, Primefaces, JSF
				\item Linux, Bash, Apache, MySQL, PostgresSQL, Samba
				\item Gradle, Gulp, Grunt, GIT, Jenkins, Subversion
			\end{ecvitemize}
		}

		\ecvblueitem{Projekt Management}{
			\begin{ecvitemize}
				\item Team orientiert
				\item Erfahrung in diversen und multikulturellen Arbeitsumgebungen im In- und Ausland
				\item Beratung und Unterstützung für Mitarbeiter im Team
				\item Anforderungsanalyse und Workload-Management basiert auf Agile-Methoden
			\end{ecvitemize}
		}

		\ecvblueitem{Führerschein}{B}

		\ecvsection{Interessen}
		\ecvitem{}{
			\begin{ecvitemize}
				\item Kraftsport
				\item Chormusik
				\item Geschichte
				\item Open-Source Software
			\end{ecvitemize}
		}

	\end{mypasscv}
\end{document}
